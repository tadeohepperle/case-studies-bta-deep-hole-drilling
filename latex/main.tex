\documentclass[12 pt]{scrartcl}
\usepackage{setspace}
\usepackage{url}
\onehalfspacing
\usepackage{amsmath,amssymb,amsfonts,amsthm,mathtools}
\usepackage[english]{babel}
\usepackage[T1]{fontenc}
\usepackage[utf8x]{inputenc}
\usepackage{lmodern}
\usepackage{dsfont}
\usepackage{bbm}
\usepackage[round]{natbib}
\usepackage{color} 
\usepackage[defaultlines=2,all]{nowidow}
\usepackage{caption}
\usepackage[labelformat=simple]{subcaption}
\usepackage{makecell}
\renewcommand\thesubfigure{(\alph{subfigure})}

\setlength\parindent{0pt}
\setlength{\parskip}{6pt plus 1pt minus 1pt}

\newcommand{\red}{\textcolor{red}}


\begin{document}

\begin{titlepage}
  \centering
  {\scshape\LARGE TU Dortmund \par}
  \vspace{1cm}
  {\scshape\Large Case Studies \par}
  \vspace{2cm}
  {\huge\bfseries Project 2: BTA Deep Hole Drilling\par}
  \vspace{2cm}
  {\Large Lecturers:\\
    Dr.\ Uwe Ligges \\
    M.\ Sc.\ Marie Beisemann\\
    M.\ Sc.\ Leonie Schürmeyer \par}
  \vspace{1cm}
  {\Large Author: Tadeo Hepperle \par}
  \vspace{0.5 cm}
  {\Large Group number: 5\par}
  \vspace{0.5 cm}
  {\Large Group members: Lennard Heinrigs, Tadeo Hepperle, Joshua Oehmen, Vanlal Peka}
  \vfill
  {\large \today\par}
\end{titlepage}

\tableofcontents

\cleardoublepage

\section{Introduction}

% todo: explain what variables are present in the data

BTA deep hole drilling is a machining process that differs from traditional boring processes in some ways.
BTA stands for Boring and Trepanning Association and describes dilling processes in which the hole depth to diameter ratio is particularly large. BTA drilling machines can create holes of a diameter of 7mm up to 700 mm and can achieve a depth to diameter radius of up to 400:1
% https://unisig.com/information-and-resources/what-is-deep-hole-drilling/what-is-bta-drilling/
Because the holes are XXXX the boring head needs



% boring head usually self guiding. that is why it is a duynamic progress with distrubances
% https://en.wikipedia.org/wiki/Deep_hole_drilling

% How does BTA work?
% what materials are drilled in?? mainly metal???
% What is BTA used for?

% How does it differ from traditional drilling?

% We got measured data of XXX boring processes and want to see if we can predict chatter before onset.
% Goal: detect the onset of chattering before it happens.

% What variables did we measure?

% How was the data recorded, gx1, format, etc.
% different variables in the different chatter vs non chatter
% what is this gx1 machine???

% What attempts have been made in research to model these time series??

% just audio does not work well (WHO said tha XXXXXX)
% - neural networks
% - area under ACF and KOlgomerov smirnov
% - frequency based????


\section{Methods}

% For reading the data, I developed a package that can read any gx1 time series. Availbale here: XXXXX
% python was used for all data analysis. These packages:  TAKE FROM REPORT 1

\subsection{Power Spectrum}

% Fourrier transform????

\subsection{ACF (Auto-Correlation-Function)}

% Autocovariance function

% Autocorrelation function

\subsection{Absolute Area under the ACF}

\subsection{Kolmogorow-Smirnow-Test}

\section{Data Analysis}

% for each of the drillings, show all known parameters from metadata: in table?

% how to detect the chatter -> Listen to all. Time series with chatter and without chatter

%  show plots with audio signal and the sections marked in the signal

% some descriptive analysis things. How long were all the time series?

\subsection{Variable exploration}

% Looking at frequency bands in the different regions??? -> Windows???

% Conclusion that torque makes most sense as a predictor.

% Comparing Torque ACF before the chattering and when the chattering started.

\subsection{Predictions}

% using KOlgomerov and Area under ACF: 

% look at torque AunderACF in comparison to different regions for that time series

% select cutoff value

\subsection{Avoid False Positives}

% check on the non chatter data that we would actually not stop the machine.

\section{Summary and Discussion}

% this is too little data to do much. Also variables different between data.

\newpage
\addcontentsline{toc}{section}{Bibliography}
\renewcommand\refname{Bibliography}
\bibliographystyle{plainnat}
\bibliography{references}
\end{document}
